\documentclass[writeup.tex]{subfiles}
\begin{document}

			
	
\section{Part 5: Discombobulated Audio} \label{section.part5}
	Playing the audio files without doing anything just sounds horrible. However, it also sounds like something that has been slowed down. The questions, then, are:
	\begin{itemize}
		\item How much are they slowed down?
		\item In which order do they need to be played?
	\end{itemize}
	
	Luckily, the answer to the second question seems to be answered by the files themselves.
	
		\begin{figure}[H]
			\centering
			\includegraphics[scale=1]{"screenshots/audio_files"}
		\end{figure}
		
		This clearly shows that each file has a track number and a title. Hopefully this solves the ordering issue.\\
		\\
		Now, I could start up Audacity\footnote{Website: \url{http://www.audacityteam.org/}} which is a really awesome tool for messing with audio files, but I want to try out something first. Namely, I want to play the files in Windows Media Player and just crank up the play speed and see if I get anything useful out of that.\\

		\begin{figure}[H]
			\centering
			\includegraphics[width=\linewidth]{"screenshots/windows_media_player"}
		\end{figure}
		
		As it turns out, when played at 8 times speed, speech actually becomes quite understandable. Here's what the fellow is saying.
		
		\begin{lstlisting}[backgroundcolor=\color{gray!25},basicstyle=\ttfamily,columns=fullflexible,breaklines]
Father Christmas, Santa Claus. Or, as I've always known him, Jeff.
		\end{lstlisting}
		
		Now, to get the punctuations right I did have to do a Google search for the phrase. But this is also the password for the secret door inside of Santa's office.
		
	\subsection{Who is the villain behind the nefarious plot.}
		We've got a full confession here.
		
\begin{lstlisting}[backgroundcolor=\color{gray!25},basicstyle=\ttfamily,columns=fullflexible,breaklines]
<Dr. Who> The question of the hour is this: Who nabbed Santa.
<Dr. Who> The answer? Yes, I did.
\end{lstlisting}

		So there we have it, it was the Doctor himself who kidnapped Jeff.
		
	\subsection{Why had the villain abducted Santa?}
		To put it in the Doctors own words.
				
\begin{lstlisting}[backgroundcolor=\color{gray!25},basicstyle=\ttfamily,columns=fullflexible,breaklines]
<Dr. Who> Next question: Why would anyone in his right mind kidnap Santa Claus?
<Dr. Who> The answer: Do I look like I'm in my right mind? I'm a madman with a box.
<Dr. Who> I have looked into the time vortex and I have seen a universe in which the Star Wars Holiday Special was NEVER released. In that universe, 1978 came and went as normal. No one had to endure the misery of watching that abominable blight. People were happy there. It's a better life, I tell you, a better world than the scarred one we endure here.
<Dr. Who> Give me a world like that. Just once.
<Dr. Who> So I did what I had to do. I knew that Santa's powerful North Pole Wonderland Magick could prevent the Star Wars Special from being released, if I could leverage that magick with my own abilities back back in 1978. But Jeff refused to come with me, insisting on the mad idea that it is better to maintain the integrity of the universe's timeline. So I had no choice - I had to kidnap him.
<Dr. Who> It was sort of one of those days.
<Dr. Who> Well. You know what I mean.
<Dr. Who> Anyway... Since you interfered with my plan, we'll have to live with the Star Wars Holiday Special in this universe... FOREVER.  If we attempt to go back again, to cross our own timeline, we'll cause a temporal paradox, a wound in time.
<Dr. Who> We'll never be rid of it now. The Star Wars Holiday Special will plague this world until time itself ends... All because you foiled my brilliant plan.  Nice work.
		\end{lstlisting}
		
		He did \textbf{not}	like the \textit{Star Wars Holiday Special}\footnote{Wiki page at \url{https://en.wikipedia.org/wiki/Star_Wars_Holiday_Special}} and felt that kidnapping Santa was the best way to prevent it from being released.
\end{document}